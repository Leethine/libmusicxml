The numbers in the picture refer to so-called passes (compiler writing terminology), i.e. atomic components of the library that convert a representation into another. The passes are numbered in the order they were added to the library:
%\begin{adjustwidth}{-0.5cm}{-0.5cm}
\begin{center}
\footnotesize
\def \contentsWidth{0.7\textwidth}
\def \arraystretch{1.3}
%
\begin{longtable}[t]{cp{\contentsWidth}}
{Passes} & {Description} \tabularnewline[0.5ex]
\hline\\[-3.0ex]
%
\texttt{1} & reads \mxml\ data from a file or from standard input is '-' is supplied as the file name, and creates an mxmlelement tree containg the same data;
\tabularnewline

\texttt{2} & converts an mxmlelement tree into \mxml\ data. This is a mere 'print()' operation;
\tabularnewline

\texttt{3} & converts an mxmlelement tree into Guido text code, and writes it to standard output;
\tabularnewline

\texttt{4} & converts an mxmlelement tree into and MSR representation. \mxml\ represents how a score is to be drawn, while MSR represents the musical contents with great detail. This pass actually consists in two sub-passes: the first one builds an MSR skeleton containing empty voices and stanzas, and the second one the fills this with all the rest;
\tabularnewline

\texttt{5} & converts an MSR representation into an LPSR representation, which contains an MSR component build from the original MSR (pass 5). The BSR contains \lily-specific representations such as {\tt $\backslash$layout}, {\tt $\backslash$paper}, and {\tt $\backslash$score} blocks;
\tabularnewline

\texttt{6} & converts an LPSR representation into an MSR representation. There is nothing to do, since the former contains the latter as a component;
\tabularnewline

\texttt{7} & converts an LPSR representation into \lily\ text code, and writes it to standard output;
\tabularnewline

\texttt{7'} & converts an LPSR representation into \lily\ text code using \lilyJianpu, and writes it to standard output. This pass is run with {\tt xml2ly -jianpu};
\tabularnewline

\texttt{8} & converts an MSR representation into a BSR representation, which contains an MSR component build from the original MSR (pass 5). The BSR contains \braille-specific representations such as pages, lines and 6-dot cells. The lines and pages are virtual, i.e. not limited in length;
\tabularnewline

\texttt{9} & converts a BSR representation into an MSR representation. There is nothing to do, since the former contains the latter as a component;
\tabularnewline

\texttt{10} & converts a BSR representation into another one, to adapt the number of cells per line and lines per page from virtual to physical. Currently, the result is a mere clone;
\tabularnewline

\texttt{11} & converts a BSR representation into \braille\ text, and writes it to standard output;
\tabularnewline

\texttt{12} & converts an MSR representation into an mxmlelement tree;
\tabularnewline

\texttt{13} & converts an MSR representation into another one, built from scratch. This allows the new representation to be different than the original one, for example to change the score after is has been scanned and exported as \mxml\ data, or to add skip (invisible) notes to avoid the \lily\ issue \#34. For simplicity and efficiency reasons, this pass is not present as such, but 'merges' within passes 6 and 9;
\tabularnewline

\texttt{14} & converts an \msdl\ score description into an MSR representation.
\tabularnewline

\end{longtable}
\end{center}

