\lib\ is instrumented with an optionnal, full-fledged trace facility, with numerous options to display what is going on when using the library.
One can build the library with or without trace, which applies to the whole code base.

% -------------------------------------------------------------------------
\subsection{Activating the trace}
% -------------------------------------------------------------------------

Tracing is controlled by {\tt TRACING_IS_ENABLED}, defined or nor in {\tt src/oah/enableTracingIfDesired.h}:

\begin{lstlisting}[language=C++]
#ifndef ___enableTracingIfDesired___
#define ___enableTracingIfDesired___

#ifndef TRACING_IS_ENABLED
  // comment the following definition if no tracing is desired
  #define TRACING_IS_ENABLED
#endif

#endif
\end{lstlisting}

This file should be included when the trace facility is used:
\begin{lstlisting}[language=C++]
#include "enableTracingIfDesired.h"
#ifdef TRACING_IS_ENABLED
  #include "traceOah.h"
#endif
\end{lstlisting}

The files {\tt src/oah/traceOah.h/.cpp} contain the options to the trace facility itself.
They provide 

For example, {\tt xml2ly -insider -help-trace} produces:
\begin{lstlisting}[language=C++]
menu@macbookprojm > xml2ly -insider -help-trace
--- Help for group "OAH Trace" ---
  OAH Trace (-ht, -help-trace) (use this option to show this group)
    There are trace options transversal to the successive passes,
      showing what's going on in the various translation activities.
      They're provided as a help to the maintainers, as well as for the curious.
      The options in this group can be quite verbose, use them with small input data!
      All of them imply '-tpasses, -trace-passes'.
  --------------------------
    Options handling trace    (-htoh, -help-trace-options-handling):
      -toah, -trace-oah
            Write a trace of options and help handling to standard error.
            This option should best appear early.
      -toahd, -trace-oah-details
            Write a trace of options and help handling with more details to standard error.
            This option should best appear early.
    Score to voices           (-htstv, -help-trace-score-to-voices):
      -t<SHORT_NAME>, -trace-<LONG_NAME>
            Trace SHORT_NAME/LONG_NAME in books to voices.
      The 10 known SHORT_NAMEs are:
        book, scores, pgroups, pgroupsd, parts, staves, st, schanges,
        .
      The 10 known LONG_NAMEs are:
        -books, -scores, -part-groups, -part-groups-details,
        -parts, -staves, -staff-details, -staff-changes, -voices and
        -voices-details.
... ... ...
\end{lstlisting}


% -------------------------------------------------------------------------
\subsection{Trace categories}
% -------------------------------------------------------------------------

% -------------------------------------------------------------------------
\subsection{Using the trace in practise}
% -------------------------------------------------------------------------

