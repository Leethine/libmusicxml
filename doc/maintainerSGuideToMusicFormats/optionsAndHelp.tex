OAH is a powerful way of representating the options together with the corresponding help. The classical {\tt getopt*()} family of functions are not up to the task because:
\begin{itemize}
\item there is a great number of options in \formats;
\item attaching the help to the options in a clean, neat way was highly desirable;
\item more important still, the re-use of options whenever conversion passes are combined into translators could only be achieved with an object oriented handling of the options and help;
\end{itemize}

For example, {\tt -beam-all-grace-notes} controls whether beams should be added to grace notes. Here is how it is implemented and used.

First, we must determine to which internal representation or conversion pass it applied to. In this case, that is the conversion pass of an {\tt mxmlTree} to \msr.
Thus we have in {\tt passes/mxmltree2msr/mxmlTree2msrOah.cppmxmlTree2msrOah.h}:
\begin{lstlisting}[language=C++]
class EXP mxmlTree2msrOahGroup : public oahGroup

    bool                  fBeamAllGraceNotes;

    bool                  getBeamAllGraceNotes () const
                              { return fBeamAllGraceNotes; }
\end{lstlisting}

In {\tt passes/mxmltree2msr/mxmlTree2msrOah.cpp}, the option is created this way:
\begin{lstlisting}[language=C++]
void mxmlTree2msrOahGroup::initializeNotesOptions ()

  // beam all grace notes
  // --------------------------------------

  fBeamAllGraceNotes = false;

  S_oahBooleanAtom
    beamAllGraceNotesAtom =
      oahBooleanAtom::create (
        "beamagn", "beam-all-grace-notes",
R"(Add a beam to all grace notes)",
        "beamAllGraceNotes",
        fBeamAllGraceNotes);
  subGroup->
    appendAtomToSubGroup (
      beamAllGraceNotesAtom);
\end{lstlisting}

And that's it. 

The option value is checked in {\tt passes/mxmltree2msr/mxmlTree2msrTranslator.cpp}:
\begin{lstlisting}[language=C++]
void mxmlTree2msrTranslator::visitStart ( S_grace& elt )

  // should all grace notes be beamed?
  if (gGlobalMxmlTree2msrOahGroup->getBeamAllGraceNotes ()) {
    fCurrentGraceIsBeamed = true;
  }

void mxmlTree2msrTranslator::handleStandaloneOrDoubleTremoloNoteOrGraceNoteOrRest (
  S_msrNote newNote)

      // create grace notes group
      fPendingGraceNotesGroup =
        msrGraceNotesGroup::create (
          inputLineNumber,
          msrGraceNotesGroup::kGraceNotesGroupBefore, // default value
          fCurrentGraceIsSlashed,
          fCurrentGraceIsBeamed,
          fCurrentMeasureNumber);

\end{lstlisting}
