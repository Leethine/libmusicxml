The formats supported by \lib\ are:
%\begin{adjustwidth}{-0.5cm}{-0.5cm}
\begin{center}
\footnotesize
\def \contentsWidth{0.6\textwidth}
\def \arraystretch{1.3}
%
\begin{longtable}[t]{lp{\contentsWidth}}
{Format} & {Description} \tabularnewline[0.5ex]
\hline\\[-3.0ex]
%
\mxml\ & a text containg markups such as {\tt <part-list>}, {\tt <time>} and {\tt <note>};
\tabularnewline

\guido\ & a text containg markups such as {\tt $\backslash$barFormat}, {\tt $\backslash$tempo} and {\tt $\backslash$crescEnd};
\tabularnewline

\lily\ & a text containg commands such as {\tt $\backslash$header}, {\tt $\backslash$override} and {\tt $\backslash$transpose};
\tabularnewline

Jianpu \lily\ & a text containg \lily\ commands and the use of \lilyJianpu\ (\url {https://github.com/nybbs2003/lilypond-Jianpu/jianpu10a.ly}) to obtain a Jianpu (numbered) score instead of the default western notation. \lilyJianpu\ should be accessible to LilyPond for it to produce the score;
\tabularnewline

\braille\ & a text containg 6-dot cells, as described in \url {http://www.brailleauthority.org/music/Music_Braille_Code_2015.pdf};
\tabularnewline

\msdl\ & a text describing a score in the \msdl\ language.
\tabularnewline

%\midi\ & binary data containg markups such as {\tt <part-list>}, {\tt <time>} and {\tt <note>};
%\tabularnewline

\end{longtable}
\end{center}

The representations used by \lib\ are:
%\begin{adjustwidth}{-0.5cm}{-0.5cm}
\begin{center}
\footnotesize
\def \contentsWidth{0.6\textwidth}
\def \arraystretch{1.3}
%
\begin{longtable}[t]{lp{\contentsWidth}}
{Representation} & {Description} \tabularnewline[0.5ex]
\hline\\[-3.0ex]
%
MSR & Music Score Representation, in terms of part groups, parts, staves, voices, notes, etc. This is the heart of the multi-language translators provided by \lib;
\tabularnewline

mxmlelement tree & a tree representing the \mxml\ markups such as {\tt <part-list>}, {\tt <time>} and {\tt <note>};
\tabularnewline

LPSR & LilyPond Score Representation, i.e. MSR plus LilyPond-specific items such as {\tt $\backslash$score} blocks;
\tabularnewline

BSR & Braille Score Representation, with pages, lines and 6-dots cells;
\tabularnewline

MDSR & MIDI Score Representation, to be designed.
\tabularnewline

%\texttt{RandomMusic} & generates an mxmlelement tree containing random music and writes it as \mxml
%\tabularnewline
%
%tools & a set of other demo programs such as {\tt countnotes}, {\tt xmltranspose} and {\tt partsummary}
%\tabularnewline
%
%\texttt{toBeWritten} & should generate an MSR containing some music and write it as \mxml, \lily and \braille
%\tabularnewline

\end{longtable}
\end{center}

